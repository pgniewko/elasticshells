\documentclass[10pt,a4paper]{article}

\usepackage[english]{babel}
\usepackage{amsmath}
\usepackage{graphicx}
\begin{document}

\begin{center}
%\title{ \bf Granular Materials }
%\author{ Pawel Gniewek }
%\maketitle
\textsc{\Large Models of static friction in granular simulations}\\[1.0cm]
\textsc{\LARGE Pawel Gniewek}\\[1.0cm]
\vspace{0.5cm}
\today

%\newpage
%\tableofcontents
\end{center}

\newpage

\section{Sokolowski et al. \cite{sok92, sok93}}
Simulations were performed in 2D with round and soft spherical particles. When two particles overlap three forces appear:
\begin{enumerate}
 \item Elastic restoration force
 \item Dissipation due to inelasticity:
 \begin{equation}
  \bar{f}_{diss} = -\gamma m_i (\bar{v}_{ij}\cdot \bar{r}_{ij})\bar{r}_{ij} / \bar{r}^2_{ij}
 \end{equation}
 where $\gamma$ is phenomenological dissipation coefficient, and $\bar{v}_{ij} = \bar{v}_{i} - \bar{v}_{j}$
 \item Shear force:
 \begin{equation}
  \bar{f}_{s} = -\gamma_s m_i (\bar{v}_{ij}\cdot \bar{t}_{ij})\bar{t}_{ij} / \bar{r}^2_{ij}
 \end{equation}
 where $\gamma_s$ is shear friction coefficient, and $\bar{t}_{ij}$ is perpendicular to $\bar{r}_{ij}$
\end{enumerate}

In fact friction should be proportional to normal force (Coulomb friction), and rotation was also neglected. 
However under deviations from spherical shape rotation is strongly surpressed and (for vibrating granular matterial)
effects of normal forces on friction can be neglected. However the major argument for neglecting Coulomb friction was 
to have fewer fit parameters.

In order to describie non-spherical particles site-site model is used. In site-site model, non-spherical
grains are replaced by a collection of rigiditly glued spheres - Figure 1.
Dynamics is solved in the same way as for spherical particles, but this time also a rotation of particles is considered. 
This time, to define sliding friction new relative tangential velocity is introduced 
(i,j - grains; n,m -spheres belonging to i and j, respectively):
\begin{equation}
 \bar{v}^{tang}_{i_{n}j_{m}} = \bar{v}_{i} - \bar{v}_{j} + \bar{\Omega}_{i} \times \bar{\rho}_{i_{n}} + \bar{\Omega}_j \times \bar{\rho}_{j_{m}}
\end{equation}
where: $\bar{\Omega}_i$ is angular velocity of a grain i, $\bar{\rho}_i$ denotes the distace from the point of contact
between spheres i and j to the center of mass of the grain i.
Then sliding force is:
\begin{equation}
 \bar{f}^{slide}_{i_{n}j_{m}} = -\gamma_s m_i (\bar{v}^{tang}_{i_{n}j_{m}}\cdot \bar{t}_{i_{n}j_{m}}) \frac{ \bar{t}_{i_{n}j_{m}} } { |\bar{t}_{i_{n}j_{m}}|^2 }
\end{equation}

and rolling force is:
\begin{equation}
 \bar{f}^{roll}_{i_{n}j_{m}} = -\mu [\bar{f}_{el} + \bar{f}_{diss}] \cdot \bar{t}_{i_{n}j_{m}} \frac{ \bar{t}_{i_{n}j_{m}} } { |\bar{t}_{i_{n}j_{m}}|^2 }
\end{equation}

Then the total shear force is then:
\begin{equation}
 \bar{f}^{roll}_{i_{n}j_{m}} = [\bar{f}_{el} + \bar{f}_{diss} + \min(\bar{f}^{slide}_{i_{n}j_{m}}, \bar{f}^{roll}_{i_{n}j_{m}})]  \frac{ \bar{t}_{i_{n}j_{m}} } { |\bar{t}_{i_{n}j_{m}}| }
\end{equation}
where again static friction has been neglected for simplicity and because models usually glue particles together to impose static friction.
However sliding force and dissipating force implicityly incorporate effects of static friction.
Additionally roughnes(and by this ability to dissipate kinetic energy) 
of the box in which grains were simulated was accounted by giridly packing of particles similar to the ones used in simulation.

\section{Poschel and Buchholtz \cite{pb93, pb94, pb95}}
Many experimental results are consequences of static friction. The model by Poschel and Buchholtz 
shows that such properites may be reproduced in simulations witout inlcuding static friction explicitely,
but instead simulating nonspherical particles represented by the collection of spheres - Figure 2. 
The total force expirienced by a pair of spheres is:
\begin{equation}
 \bar{F} = F^{N} \bar{n} + F^{T} \bar{t} + F^{S} \bar{n}
\end{equation}
where (in 2D) $\bar{n}$ and $\bar{t}$ are normal and tangential vectors,

$\bar{n} = \frac{\mathbf{x}_{i} - \mathbf{x}_j}{|\mathbf{x}_{i} - \mathbf{x}_j|} $, and

$\bar{t} = \begin{pmatrix} 0 & -1 \\ 1 & 0 \end{pmatrix} \frac{\mathbf{x}_{i} - \mathbf{x}_j}{|\mathbf{x}_{i} - \mathbf{x}_j|} $

Normal force is a combination of elastic repulsion and kinetic energy dissipation:
\begin{equation}
 F^{N}_{ij} = Y(r_j + r_j - |\mathbf{x}_{i} - \mathbf{x}_j| ) - \gamma_N m_{eff} (\dot{\mathbf{x}}_{i} - \dot{\mathbf{x}}_j)\cdot \bar{n}
\end{equation}
where: $m_{eff} = \frac{m_1 m_2}{m_1 + m_2}$
$Y$ is Young modulus, and $\gamma_N$ being phenomenological friction coefficient (related to the coefficient of restitution).

Contrary to \cite{sok93} spheres are not glued rigidly but connected by springs.
This there is daping force between particles belonging to the same grain and connected by spring:
\begin{equation}
 F^{S}_{ij} =\alpha \left(C^{(k)} -  |\mathbf{x}_{i} - \mathbf{x}_j| \right) + \gamma_{Sp} \frac{m_i}{2} |\dot{\mathbf{x}}_{i} - \dot{\mathbf{x}}_j|
\end{equation}
where $\alpha$ is the spring constant, and $\gamma_{Sp}$ is the dumping coefficient. $C^{(k)}$ is the "equilibrrium" distance.
As it can be seen, interaction between particles in the same  graine are purely geometric without any intergrain friction.
The static friction should be an emergent property as a result of grains' roughness. 
An exception are spherical particles for which static friction has been included as equal to\cite{cs79, herr93, herr94}:
 
\begin{equation}
F^{T}_{ij} = -sign(\bar{\nu}_{rel}) \min(\gamma_{s} m_{eff} |\bar{\nu}_{rel}| , \mu F^{N}_{ij} )
\end{equation}
where:
$\bar{\nu}_{rel} = (\dot{\mathbf{x}}_{i} - \dot{\mathbf{x}}_j)\bar{t} + \bar{r}_i \dot{\bar{\Omega}}_i + \bar{r}_j \dot{\bar{\Omega}}_j$,
$\dot{\bar{\Omega}}$ - angular velocity, $\gamma_s$ shear friction coefficient, $\mu$ - Coulomb coefficient.


The spatial configuration of these
spheres or polygons represents the microstructure of non-
spherical particles. No explicit static friction force need be
used for these compound non-spherical objects.


\section{Schafer et al. \cite{sw95, schafer96}}
Model in 2D. Review kind of paper.
Let's define a deformation as being:
\begin{equation}
 \xi = \max(0, R_i + R_j - |\mathbf{r}_2 - \mathbf{r}_1|)
\end{equation}
and $\mathbf{n}$ and $\mathbf{s}$ being normal and tangential vecotrs of the contact of two grains.
Then normal and shear velocities are defined as:
\begin{equation}
 \dot{\xi} = v_n = (\mathbf{v}_i - \mathbf{v}_j) \cdot \mathbf{n}
\end{equation}
and
\begin{equation}
 v_s = (\mathbf{v}_i - \mathbf{v}_j) \cdot \mathbf{s} + \omega_1 R_1 + \omega_2 R_2
\end{equation}

\subsection{Normal Impacts}
Two colliding spheres undergo a deformation which is between perfeclty elastic and inelastic. Dissipation of kinetic energy may go through
plastic deformation, visoelasticity, or exception of elastic waves. 
The elasticity of the impact is 
phenomenologically described by coefficient of normal resustitution being defined as $\epsilon = -v^{after}_n/v^{before}_n$. It is important
to remember that $\epsilon$ depends on the particular case of collision but may also depend on the velocity of impact.
For two elastic materials, plastic deformation may occur for the object with the velocity of impact higher of equal to:
\begin{equation}
 v^2_{yield} \approx 107 \frac{R^3_{eff} Y^5}{m_{eff} E^4_{eff}}
\end{equation}
where : $m_{eff} = \frac{m_1m_2}{m_1 + m_2}$, $R_{eff} = \frac{R_1R_2}{R_1 + R_2}$, and $\frac{1}{E_{eff}} = \frac{1-\nu^2_1}{E_1} +\frac{1-\nu^2_2}{E_2}$; 
$\nu$ being Poisson's ratio. Below that limit no plastic deformation occurs, and the energy is dissipated in a viscoelastic way.
Modeling such force requires two components: repulsion and some sort of dissipation. The simplest form is damped harmonic oscilator:
\begin{equation}
 F_{n} = \min(0,-k_n\xi - \gamma_n \dot{\xi})
\end{equation}
where: $k_n$ is material stiffnes, and $\gamma_n$ is a damping coefficient. Min funciton is added so the force is always repulsive,
because $-k_n\xi - \gamma_n \dot{\xi}$ may become positive near the end of a collision.
For Hertz theory this equation reads:
\begin{equation}
 F_{n} = \min \left(0, -\frac{4 \sqrt{R_{eff}} E_{eff}}{3} \xi^{3/2} \right)  =  \min(0, -\hat{k}_{n}\xi^{3/2})
\end{equation}
which can be extendet do dissipative Hertz by adding term: $- \gamma_n \dot{\xi}$, however this model leads to the increase of
elasticity(i.e. $\epsilon$) as the velocity of impact increases which is contrary to experiments, which can be corrected by 
introducing term: $ -\hat{\gamma}_n \xi^{1/2} \dot{\xi}$.



\subsection{Frictinal contacts}
Static friction for $v_s = 0$:
\[
 F_s \le \mu_s F_n
\]
and dynamic friction for $v_s \ne 0$:
\[
 F_d = \mu_d F_n
\]

For elastically similar bodies shear stress do not modify normal force so Hertz model can be still applied. 
The simplies dynamic shear force applies Coulomb rule \cite{haff86}:
\[
 F_s = -\mu |F_n| sign(v_s)
\]
Obviously it cannot provide reversal of tangential velocity, it can only slow down $v_s$ to zero.
However this model has discontinouity at $v_s = 0$ which in rolling regime ($v_s \Rightarrow 0$) gives jumps in the sign of 
$F_s$ instead of just 0.

Sometimes viscous friciton is also introdued \cite{sok92b}:
\[
 F_s = -\gamma_s \nu_s
\]
where: $\gamma_s$ is shear damping constant with no clear physical meaning but it should provide values high enough for collisional properties.
The problem is that when the force is nearly or larger the normal force
model provides incorrect description (since it is not limited by $F_n$).
Problems of both models can be overcome by combining them to get\cite{pb93, pb94, pb95}:
\[
 F_s = -\min(|\gamma_s \nu_s|,|\mu F_n|) \cdot sign(\nu_s)
\]

However the problem is that these model do not account tangential elasticity. 
he shear friction forces discussed above can only slow
movement in the tangent direction, not stop or reverse it.
These rules are inadequate for applications that require truly
static friction, such as heap formation, angles of repose, and
the quasi-static stages of avalanches. In such situations, there
is a threshold force below which the grains do not move at
all, opposed by static friction originating in the even smaller
scale surface interactions of particle contact.However, im-
plementation of a simple threshold rule is history dependent
and computationally intensive. 


\subsection{Tangential elasticity}
TE first introuced by Cundell and Stuck \cite{cs79}, providing very accurate results and defined as:
\begin{equation} \label{eq:fs}
 F_s = -\min(|k_s \xi|,|\mu F_n|) \cdot sign(\xi)
\end{equation}
where: $k_s$ is some tangential elasticity, and $\xi$ is the total dispalcement in a tangential direction that took place
since the moment of contact creation(at $t_0$):
\[
 \xi(t) = \int^{t}_{t_0} \nu_s(t')dt'
\]
A naive implementation of these equations can lead to some unphysical behavior,
since long-lasting contacts may involve several changes from rolling to sliding and
back, without breaking contact between the disks. This in turn can produce arbitrarily large values for $\xi$, 
since by definition it keeps on growing even when the
disks are sliding. 
The second difficulty is that representing static friction with a fictitious spring has the disadvantage 
of introducing tangential oscillations in the
system. In order to damp them one needs to put some dissipation in this tangential
interaction. Following the linear-spring dash-pot model one needs to add to \ref{eq:fs}
a term of the form $\gamma_s \dot{\xi}$

\cite{perez}

The cotinuation of the tangential elasticity model is the model introducing force dependend elastcity \cite{wb86}:
\[
 F_s(t+1) = F_s(t) + k_s(F_s(t), F_n(t), \Delta \xi(t)) |\xi(t-1) - \xi(t)|
\]
where $t$ denotes "time steps" performed in integration scheme till break of the contact. 
In general $k_s$ incorporates information on initial parameters of the contact, directionality of a deformation of the contact,
and normal deformation of the grain.

\section{Mucha et al \cite{mucha05}}
In order to draw upon the existing
theory for spherical particles, we model non-spherical parti-
cles with collections of spheres, avoiding overly stiff inter-
actions by replacing internal springs with rigid motion con-
straints. With static friction obtained via normal forces be-
tween compounds of spheres, we can then limit our attention
to dynamic shear forces.

In such model The net force
on a grain is simply the sum of the individual contact forces
applied to its particles.

\begin{figure}[tb]
\centering
\includegraphics[width=0.8\textwidth]{./pics/shapes.png}
\caption{abc}
\label{fig:shapes}
\end{figure}

\begin{figure}[tb]
\centering
\includegraphics[width=0.8\textwidth]{./pics/camel.png}
\caption{abc}
\label{fig:camel}
\end{figure}




\thispagestyle{empty} % No slide header and footer

\bibliographystyle{unsrt}
\bibliography{article}

\clearpage

\end{document}

%-Cruavzte ssh --progress --no-whole-file --stats --sparse --exclude *.svn*

