\documentclass[10pt,a4paper]{article}

\usepackage[english]{babel}
\usepackage{amsmath}
\usepackage{graphicx}
\begin{document}

\begin{center}
%\title{ \bf Granular Materials }
%\author{ Pawel Gniewek }
%\maketitle
\textsc{\Large Models of dry friction in granular simulations}\\[1.0cm]
\textsc{\LARGE Pawel Gniewek}\\[1.0cm]
\vspace{0.5cm}
\today

%\newpage
%\tableofcontents
\end{center}

\newpage

\section{Definitions}
Following definitions will be used in the later sections:

Grains deformation:
\begin{equation}
 \xi = \max(0, R_i + R_j - |\mathbf{x}_2 - \mathbf{x}_1|)
\end{equation}

Vector normal to the contact surface:
\begin{equation}
 \mathbf{n} = \frac{\mathbf{x}_{i} - \mathbf{x}_j}{|\mathbf{x}_{i} - \mathbf{x}_j|}
\end{equation}
and tangent vector (in 2D):
\begin{equation}
 \mathbf{t} = \begin{pmatrix} 0 & -1 \\ 1 & 0 \end{pmatrix} \frac{\mathbf{x}_{i} - \mathbf{x}_j}{|\mathbf{x}_{i} - \mathbf{x}_j|}
\end{equation}


Normal and shear velocities are defined as:
\begin{equation}
 \dot{\xi} \equiv v_n = (\mathbf{v}_i - \mathbf{v}_j) \cdot \mathbf{n}
\end{equation}
and
\begin{equation} \label{eq:vs1}
 \mathbf{\nu}_s = (\mathbf{v}_i - \mathbf{v}_j) \cdot \mathbf{t}
\end{equation}
or in more general form:
\begin{equation}  \label{eq:vs2}
 \mathbf{\nu}^{rot}_s = (\mathbf{v}_i - \mathbf{v}_j) \cdot \mathbf{t} + \Omega_1 \times \mathbf{r}_1 + \Omega_2 \times \mathbf{r}_2
\end{equation}
where $\mathbf{r}_1$ and $\mathbf{r}_2$ go from the center of the grain to the point of the contact, and $\Omega$ is an angular velocity vector.

\section{Forces}
\begin{enumerate}
 \item Normal force account for elastic deformation and energy dissipation, which in general is:
 \begin{equation}\label{eq:normal}
  \mathbf{F}^{n} = \mathbf{F}^{el} + \mathbf{F}^{d} = -k_n\mathbf{\xi}^{\alpha} - \gamma_n \mathbf{\xi}^{\beta}\dot{\mathbf{\xi}}
 \end{equation}


 \item Tangential forces: $\mathbf{F}^{s}$
 \item Fine structure forces: $\mathbf{F}^{f}$ - if grains have sub-structure.

\end{enumerate}


\section{Sokolowski et al. \cite{sok92, sok93}}
In Sokolowski's model normal force takes the form of equation \ref{eq:normal} with $\alpha=1.0$, and $\beta=0.0$.
Then using equation \ref{eq:vs1} shear force is given by:

\begin{equation}
  \mathbf{F}^{s}_i = -\gamma_s m_i \mathbf{\nu}_s
\end{equation}

However this model provides incorrect behavior for large velocities since shear force should be restricted 
by normal force (Coulomb friction). Moreover grains rotation has been also neglected. 
However for non-spherical grains in dense systems rotation and the effects of normal forces on friction are strongly suppressed. 

For the case where non-spherical particles are needed the so called "site-site" model is used. In site-site model, non-spherical
grain is represented as a collection of rigidly glued spheres - Figure \ref{fig:shapes}.
Dynamics is solved in the same way as for spherical particles but for this case rotation is also taken into account:

\begin{equation}
 \mathbf{F}^{s}_{ij} = \left[\mathbf{F}^{n}_i + \min(\mathbf{F}^{slide}_{ij}, \mathbf{F}^{roll}_{ij})\right]  \frac{ \mathbf{t}_{ij} } { |\mathbf{t}_{ij}| }
\end{equation}

where sliding force for spheres \textit{i} and \textit{j} belonging to grains two different grains is: %\textit{n} and \textit{m} is:
\begin{equation}
 \mathbf{F}^{slide}_{ij} = -\gamma_s m_i \nu^{rot}_s
\end{equation}
and rolling force is:
\begin{equation}
 \mathbf{F}^{roll}_{ij} = -\mu\mathbf{F}^{n}_{i} \cdot \mathbf{t}_{ij} \frac{ \mathbf{t}_{ij}} { |\mathbf{t}_{ij}|^2 }
\end{equation}

Again static friction is neglected. % for simplicity and because models usually glue particles together to impose static friction.
However sliding and rolling forces implicitly imitate static friction for grains with intrinsic roughness.
%Additionally roughness (and by this ability to dissipate kinetic energy) 
%of the box in which grains were simulated was accounted by rigidly packing of particles similar to the ones used in simulation.

\begin{figure}[tb]
\centering
\includegraphics[width=0.8\textwidth]{./pics/shapes.png}
\caption{On the left: tetrahedron is represented as 4 rigidly connected spheres. On the right: cube is represented as 9 spheres of different size.
Static friction is a result of geometric roughness of the surface (generating spatial hindrance) and dynamic friction \cite{sok93, mucha05}.}
\label{fig:shapes}
\end{figure}

\section{Poschel and Buchholtz \cite{pb93, pb94, pb95}}
Despite the fact that many properties of non-spherical granular materials are consequences of static friction,
Poschel and Buchholtz \textbf{show that such properties can be reproduced in simulations without explicitly including static friction.}
Non-spherical grains are represented as a collection of interacting spheres - Figure \ref{fig:pb}. 
The total force experienced by a pair of spheres is:
\begin{equation}
 \mathbf{F} = \mathbf{F}^{n} + \mathbf{F}^{s} + \mathbf{F}^{f}
\end{equation}

Contrary to the representation in Figure \ref{fig:shapes}, spheres are not glued rigidly but connected by elastic springs - Figure \ref{fig:pb}.
To prevent sub-structural vibrations damping component is also included:
\begin{equation}
 |\mathbf{F}^{f}_{ij}| =k_\alpha \left(C_{ij} -  |\mathbf{x}_{i} - \mathbf{x}_j| \right) + \gamma_{f} \frac{m_i}{2} |\dot{\mathbf{x}}_{i} - \dot{\mathbf{x}}_j|
\end{equation}
where $k_\alpha$ is the spring constant, and $\gamma_{f}$ is the dumping coefficient. $C_{ij}$ is the \textit{equilibrium} distance between spheres i and j
which belong to the same grain.
%As it can be seen, interaction between particles in the same  grain are purely geometric without any inter-grain friction.
The static friction should be an emergent property as a result of grains' roughness. 
An exception are spherical particles for which static friction has been included as \cite{cs79, herr93, herr94}:
 
\begin{equation}
\mathbf{F}^{s}_{i} = - \min(\gamma_{s} \tilde{m} |\mathbf{\nu}^{rot}_s| , \mu \mathbf{F}^{n}_{i} ) \frac{\mathbf{\nu}^{rot}_s}{|\mathbf{\nu}^{rot}_s|}
\end{equation}
where:
$\gamma_s$ kinetic friction coefficient, $\mu$ - static friction coefficient, and $\tilde{m} = \frac{m_1m_2}{m_1+m_2}$


%The spatial configuration of these spheres or polygons represents the heterostructure of non-spherical particles. 
%No explicit static friction force need to be used for these compound non-spherical objects.

\begin{figure}[tb]
\centering
\includegraphics[width=0.8\textwidth]{./pics/pb.png}
\caption{Square represented as a collection of 5 spheres connected by (relatively stiff) harmonic springs. Roughness of the edges imitate static friction of the square \cite{pb93}.}
\label{fig:pb}
\end{figure}

\section{Schafer et al. \cite{sw95, schafer96}}
\subsection{Normal Impacts - limits of an application of elasticity theory}
Two colliding grains undergo a deformation which is between perfectly elastic and inelastic. Dissipation of kinetic energy may go through
plastic deformation, viscoelasticity, or excitation of elastic waves. 
The elasticity of the impact is phenomenologically described by coefficient of restitution being defined as $\epsilon = -v^{after}_n/v^{before}_n$. 
%It is important to remember that $\epsilon$ depends on the particular case of collision but may also depend on the velocity of impact.
For two elastic materials, plastic deformation may occur for the object with the velocity of impact equal or larger than:
\begin{equation} \label{eq:impact}
 v^2_{yield} \approx 107 \frac{\tilde{R}^3 Y^5}{\tilde{m} \tilde{E}^4}
\end{equation}
where : $\tilde{m} = \frac{m_1m_2}{m_1 + m_2}$, $\tilde{R} = \frac{R_1R_2}{R_1 + R_2}$, and $\frac{1}{\tilde{E}} = \frac{1-\nu^2_1}{E_1} +\frac{1-\nu^2_2}{E_2}$; 
$\nu$ being Poisson's ratio. Below this limit no plastic deformation occurs and the energy is dissipated in a viscoelastic way.
Modeling such force requires two components: repulsion and some sort of dissipation. The simplest form is damped harmonic oscillator is in the spirit of equation \ref{eq:normal}:
\begin{equation}
 F_{n} = \min(0,-k_n\xi - \gamma_n \dot{\xi})
\end{equation}
where: $k_n$ is material stiffens, and $\gamma_n$ is a damping coefficient. Min function is added so the force is always repulsive -
since $-k_n\xi - \gamma_n \dot{\xi}$ may become positive near the end of a collision.
For Hertz theory this equation reads:
\begin{equation}
 F_{n} = \min \left(0, -\frac{4 \sqrt{R_{eff}} E_{eff}}{3} \xi^{3/2} \right)  =  \min(0, -\hat{k}_{n}\xi^{3/2})
\end{equation}
which can be extended do dissipative Hertz by adding term: $-\gamma_n \dot{\xi}$, however this model leads to the increase of
elasticity(i.e. $\epsilon$) as the velocity of impact increases which is contrary to experiments - 
can be corrected by introducing term: $ -\hat{\gamma}_n \xi^{1/2} \dot{\xi}$.



\subsection{Frictional contacts}
Static friction for $\nu_s = 0$:
\begin{equation}
 \mathbf{F}^s \le \mu_s \mathbf{F}^n
\end{equation}
and dynamic friction for $v_s \ne 0$:
\begin{equation}
 \mathbf{F}^s = \mu_s \mathbf{F}^n
\end{equation}

For similar bodies (size and elasticity) shear stress do not modify normal force so Hertz model can be still applied. 
The simplest dynamic shear force applies Coulomb rule \cite{haff86}:
\begin{equation}
 \mathbf{F}^s = -\mu |\mathbf{F}^n| \frac{\nu_s}{|\nu_s|}
\end{equation}
Obviously it cannot provide reversal of tangential velocity - it can only slow down $v_s$ to zero.
Moreover this model has discontinuity at $v_s = 0$ which in rolling regime ($v_s \rightarrow 0$) gives jumps in the sign of 
$F_s$ instead of just 0.

Sometimes viscous friction is also introduced \cite{sok92b}:
\begin{equation}
 \mathbf{F}^s = -\gamma_s \nu_s
\end{equation}
where: $\gamma_s$ is shear damping constant. % with no clear physical meaning but it should provide values high enough for collisional properties.
The problem is that when the force is nearly the same or larger than the normal force,
model provides incorrect description (since it is not limited by $\mathbf{F}^n$).
These problems can be overcome by \cite{pb93, pb94, pb95}:
\begin{equation} \label{eq:stat1}
 \mathbf{F}^s = -\min(|\gamma_s \nu_s|,|\mu \mathbf{F}^n|) \cdot \frac{\nu_s}{|\nu_s|}
\end{equation}

Finally the problem is that these models do not account for tangential elasticity. 
The shear friction forces discussed above can only slow down the motion in the tangential direction.
These rules are inadequate for applications that require truly static friction, such as heap formation, angles of repose, and
the quasi-static stages of avalanches.  In such situations, there is a threshold force below which the grains do not move at  all, 
opposed by static friction originating in the even smaller scale surface interactions of particle contact. 
However implementations of a simple threshold rule are history dependent and computationally expensive. 


\subsection{Tangential elasticity}
Tangential elasticity was first introduced by Cundell and Stuck \cite{cs79}, providing very accurate results and is given as:
\begin{equation} \label{eq:fs}
 \mathbf{F}^s = -\min(|k_s \xi_s|,|\mu \mathbf{F}^n|) \cdot sign(\xi_s)
\end{equation}
where: $k_s$ is tangential elasticity, and $\xi_s$ is the total displacement in a tangential direction that took place
since the moment of a contact creation(at $t_0$):
\begin{equation}
 \xi_s(t) = \int^{t}_{t_0} \nu_s(t')dt'
\end{equation}
The interpretation of this model is a \textbf{transverse spring introduced at the position of contact}, which oppose deformation of the contact.
A naive implementation of these equations can lead to some unphysical behavior,
since long-lasting contacts may involve several changes from rolling to sliding and
back, without breaking contact between the disks. This in turn can produce arbitrarily large values for $\xi$, 
since by definition it keeps on growing even when the
disks are sliding. 
The second difficulty is that representing static friction with a fictitious spring has the disadvantage 
of introducing tangential oscillations in the
system. In order to damp them one needs to put some dissipation in this tangential
interaction. Following the linear-spring dash-pot model one needs to add to equation \ref{eq:fs}
a term of the form $\gamma_s \dot{\xi}$ \cite{perez}.

%The continuation of the tangential elasticity model is the model introducing force depended elasticity \cite{wb86}:
%\begin{equation}
% \mathbf{F}^s(t+1) = \mathbf{F}^s(t) + k_s(\mathbf{F}^s(t), \mathbf{F}^n(t), \Delta \xi_s(t)) |\xi_s(t-1) - \xi_s(t)|
%\end{equation}
%where $t$ denotes "time steps" performed in integration scheme till break of the contact. 
%In general $k_s$ incorporates information on initial parameters of the contact, directionality of a deformation of the contact,
%and normal deformation of the grain.

\section{Mucha et al \cite{mucha05}}
In order to draw upon the existing theory for spherical particles, non-spherical particles are modeled with collections of spheres - Figure \ref{fig:camel}, 
avoiding overly stiff interactions by replacing internal springs by rigid constraints. 
With static friction obtained via normal forces between compounds of spheres, shear model is reduced to dynamic shear forces.
In such a model the net force acting on a grain is simply the sum of the individual contact forces applied to its all spheres,
and static friction is the results of roughness of she surface. \textbf{Thus as a result of geometric hindrances static friction emerges implicitly.}

\begin{figure}[tb]
\centering
\includegraphics[width=0.8\textwidth]{./pics/camel.png}
\caption{Camel represented as a set of small spheres\cite{mucha05}. Other objects may interact with the camel by individual interaction with each sphere separately.
Friction is then calculated using equation \ref{eq:stat1}.}
\label{fig:camel}
\end{figure}

\section{Comments}
\begin{enumerate}
 \item It seems that the friction presented above is mainly applied to macroscopic objects ($\sim 1[mm]$), where the continuity assumption holds. 
 \item For our considerations impact velocities are so small that we can neglect possibility of plastic deformation given by \ref{eq:impact}.
 \item In some cases introducing static friction has not been necessary in order to reproduce physical properties of for example sand heaps \cite{pb93, pb95}.
 \item For transverse spring given by equation \ref{eq:fs} a history of the contact must be monitored. 
 Moreover the condition for breaking a contact must be additionally introduced. 
 \item Non-spherical objects need a fine representation, which usually goes through covering a surface by spheres. This representation 
 introduces roughness which is, in many instances, enough for reproducing static friction-like behavior on large scales - 
 however short-range behavior(near the place where the contact is created) may be incorrect.
 \item Assuming molecular scale being $L_{mol}\approx 0.5 [nm]$, and accuracy resolution $\epsilon=0.001$, length-scale for which density flucs' can be neglected is:
 $L_{micro} = \epsilon^{-2/3} L_{mol} = 50 [nm]$, and the length-scale at which macroscopic smoothness of the material allows us to introduce phenomenological friction
 parameter is $L_{macro} = \epsilon^{-1} L_{micro} = 50[\mu m]$, which is slightly larger than the dimension of mammalian cell. Thus if we want to introduce even larger resolution 
 to the description of yeast cell than the frictional(cont. mech.) description may not be a proper one.
 \item For systems below $L_{macro}$ length-scales,
 \textbf{friction at large scales may emerge from the adhesion of elements making up a cell wall and non-smooth representation of cellular surface.}
 \item Based on calculations presented in inertia.pdf it seems that cells motion is significantly over-damped.
 Because of this probably cells would not significantly move even in the absence of other cells, and thus equations \ref{eq:stat1} or  \ref{eq:fs} give us the same answer.
 %\item Upon high compression (where cell wall can be burst)and assuming even relatively small friction coefficient $\nu \approx 0.1$ we see that 
 %the magnitude of friction force (assuming it is valid to define such) may be of the order of $F = $.
 \item The general theory of friction is still unknown, so in fact we do not know how macroscopic friction emerges from the microscopic properties of material \cite{kess01}.
 \item Friction coefficients for corneal cells are of the order $\mu=0.03-0.06$ \cite{ang12}.
\end{enumerate}




\thispagestyle{empty} % No slide header and footer

\bibliographystyle{unsrt}
\bibliography{article}

\clearpage

\end{document}

%-Cruavzte ssh --progress --no-whole-file --stats --sparse --exclude *.svn*

