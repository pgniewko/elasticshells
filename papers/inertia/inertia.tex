\documentclass{article}
\usepackage{graphicx}

\begin{document}

\title{Role of the inertial terms}

\maketitle

Single cell which is moving through liquid and experiencing an external force $F_{ext}$, the equation of motion is:
\begin{equation}\label{eq:force}
  m \frac{d^2 r}{dt^2} = F_{ext} - f \frac{d r}{d t}
\end{equation}
where: m - mass, f - friction coefficient.
Switching to dimensionless quantities: $s = r/L$, and $\tau=t/T$ equation \ref{eq:force} leads to:
\begin{equation}
 \frac{m}{f T} \frac{d^2 s}{d \tau^2} = \frac{T}{f L} F_{ext} - \frac{ds}{d \tau}
\end{equation}

Now let's take: $L$ being a typical size of a cell $L=10 \mu m$, $m=\rho L^3$, where $\rho = 1g/cm^3$.
Assuming spherical shape of a cell: $f = 6 \pi \eta r$, where $r=L/2$, we can write: $f/L \approx \eta$.
Thus the coefficient of dimensionless acceleration term becomes: $\rho \nu L / \eta$, where $\nu = L/T$. 
Taking a characteristic time ranging from 1s -  1000s, and $\eta=0.001 Pa \cdot s$ (i.e. water) we obtain:
$\rho \nu L / \eta \approx 10^{-7} - 10^{-4}$, which makes this term negligibly small in comparison to
$\frac{d s}{d t}$.

On the other hand if we neglect $F_{ext}$, we obtain:

\begin{equation}
 m \frac{d v}{d t} = -f v
\end{equation}
and its solution:
\begin{equation}
 v(t) = v(0) e^{-\frac{t}{\lambda}}
\end{equation}
with a characteristic time-scale $\lambda=m/f$. Plugging in all the numbers we obtain $\lambda=0.1 \mu s$ 
which means that the motion ceases almost instanlty after the perturbation.
In conclusion acceleration term can be neglected in equations of motion, and cell only moves in a response to external forces. 
Additinally the total distance the bacterium coasts is:
\begin{equation}
 d = \int_{0}^{\infty} \nu(t) dt = \int_{0}^{\infty} \nu(0)e^{-\frac{t}{\tau}} dt = -\nu(0) e^{-t/\tau} | ^{\infty}_{0} = \nu(0) \tau = \frac{m}{f}  \nu(0)
\end{equation}

which for $\nu(0)=25 [\mu m] [s^{-1}]$ would give $d = 5 \times 10^{-6} \mu m$ which is negligible even on the scale of bacterium.



\end{document}