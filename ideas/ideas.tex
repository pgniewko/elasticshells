\documentclass[10pt,a4paper]{article}

\usepackage[english]{babel}


\begin{document}

\begin{center}
%\title{ \bf Granular Materials }
%\author{ Pawel Gniewek }
%\maketitle
\textsc{\Large Granular Materials}\\[1.0cm]
\textsc{\LARGE Pawel Gniewek}\\[1.0cm]
\vspace{0.5cm}
\today

\newpage
\tableofcontents
\end{center}

\newpage

\section{Ideas}
\subsection{Simulations based on Scherck's papers}
\begin{enumerate}
 \item Simulate growing balls with some sort of repulsion: $V = \frac{K}{2}(r-\sigma)^{\alpha}$, where $\alpha=2.5$ is a Hertzian interaction.
 The reasoning behind $\alpha >2.0$ is that when two balls collide and deform one another, 
 the  interface between them, grows faster then the distance separating then gets smaller.
 \item $\alpha=2,\frac{5}{2},0$ - harmonic, Hertzian(in 3D), and hard-spehre repulsion.
 \item Simulate the box with a periodic boundary conditions. However if you want to introduce walls into your box
 do it by the interactions with mirror images (in the respect of the position of the imaginary wall). Be carefull
 around simulating the hole where balls can escape. 
 \item Seed the box with balls' seeds, and then grow them to the desired density.
 Monitor energy of the system. Calculate forces, and then if available relax the system: grow at the beginning and bisecting 
 \item Use different size balls, let's say the radius ration $r_1/r_2 \approx 1.4$
 \item What is boson peak ?
 
\end{enumerate}

\section{Papers}
\subsection{Jamming is not just cool any more\cite{nagel:98}}
\begin{enumerate}
 \item Jammed things: traffic, grainsand, powder raw material.
 \item vibrations from the pounding can reinitiate flow.
 \item When jammed, the disordered system is caught in a small region of phase space with no possibility to escape
 \item Jammed systems are fundamentally different. If the applied stress changes the structure by even a little, the whole jam breaks up. 
 \item One example of a system that is jammed and yet not fragile is foam.
 \item supercooled water, which freezes into glassy state can also be considered as jammed system - sampling noneq. limited space.
 \item Jamming can occure when density is high enough. 
 \item One then can unjam the system by rising the temperature(like in the case of supercooled water), or by applying stress. 
 \item ``Temperature`` - any kind of vibration (mechanical or thermal).
 \item Glass may have a lower glass transition temperature under high shear stress. 
 Likewise, a jammed granular material or foam may have a lower yield stress when random motions (that is, thermal fluctuations) are present.
 \item Conceptual framework from which to investigate the jamming transition, 
 suggesting a possible analogy between temperature and shear stress, the letter being considered alike a thermodynamic parameter.
 \item Examples of three axes:
 \begin{itemize}
  \item A liquid with low viscosity solidifies into a glass when temperature is lowered.
  \item Flowing foam becomes rigid when the applied stress is lowered.
  \item Colloidal suspension loses the ability to flow when the density is increased. 
 \end{itemize}

\end{enumerate}

\subsection{Highly evolved grains \cite{ohern:13}}
\begin{enumerate}
 \item They are far-from-equilibrium systems, as they are too large to experience thermal fluctuations and thus must be externally driven to induce particle motion.
 \item Packings of grains with wishbone shapes stiffen following a concave-up stress-strain curve, 
 and possess failure stresses that are more than a factor of three larger than those for spherical grains.
 \item Other relevant properties: density, shear modulus.Can be optimized as a function of interparticle forces, and packing protocol.
 \item Long-standing open task: identifying the most dilute jammed packings of spherical particles.
\end{enumerate}

\subsection{Repulsive Contact Interactions Make Jammed Particulate Systems Inherently Nonharmonic\cite{schreck:11}}
\begin{enumerate}
 \item When isostatic(in which the number of contacts is the minimum required for mechanical stability) systems lose even a single contact, they become fluidized.
 \item One-side repulsive potential make isostatic jammed system nonharmonic even in the limit of vanishing perturbation.
 \item Isostatic jammed system becomes nonharmonic when even single conact is borke. It occurs above the amplitude $\delta_c$, which average over all exctiation modes $\left<\delta_c\right>$ tends to zero with the number of particles
 \item Interactions potential: $V(r_{ij}) = \frac{\epsilon}{2} \left( 1 - \frac{r_{ij}}{\sigma_{ij}} \right)^2 \Theta\left( 1 - \frac{r_{ij}}{\sigma_{ij}} \right)$, where: $\sigma_{ij} = (\sigma_i + \sigma_j)/2$
 \item The MS(mechanically stable) packings were generated using the compression and energy minimization protocol.
 \item In nonharmonic regime the density of vibrational modes cannot be described using the dynamical matrix ($U_{x,y} = \partial^2 V / \partial x\partial y$)
\end{enumerate}

\subsection{Recent results on the jamming phase diagram \cite{con:10}}
\begin{enumerate}
 \item Soft, frictionless spheres, and jamming defined as a divergence in relaxation time.
 \item At zero temperature and zero shear stress (on the voluem fraction axis i.e. no particle motions), jamming is identified as appearance of mechanical strenght.
 \item The system accuaries mechanical strenth at J-point.
 \item Jamming transition at J-point has mixed first-order-second-order behaviour:
 coexistence of quantities varying continuously at the transition, such as the pressure or the shear modulus, 
 and of quantities changing discontinuously, such as the mean contact number per particle. 
 \item Jamming surface may be also considered as a surface of constant $\tau_g$, where $\tau_g$ is the longest available
 relaxation time to an experiment or simulation - but there is a debate on the shape at $\tau=\infty$.
 \item Some theoretical work suggest that true jamming for soft spheres system can occur only at T=0 (relaxation time diverge only at T=0).
 \item Fluc's in athermal system occur only when system is flowing. When system is jammed no fluc's can ever occur. This is in contrast with the
thermal case, in which fluctuations always occur, even if large fluctuations able to unjam the system may be so rare that they are not observed within any reasonable observation time.
 \item Jamming may occur not at the specific singular poing $\phi$, but in the range $\phi=0.636-0.646$. The reason is that jammed system is
 out-of-equilibrium so the behaviour may depend on the procedure. Thus jamming may occur in narrow, but still segmend of $\phi$.
 \item At $\phi=0.646$ which can be obtained by slow 'equilibrium' compression, because at thaat density pressure diverge.
\end{enumerate}

\subsection{Jamming by shear\cite{bi:11}}
\begin{enumerate}
 \item A jammed system can resist small stresses without deforming irreversibly.
 \item Jamming of frictional, disk-shaped grains can be induced by the application of shear 
 stress at densities lower than the critical value, at which isotropic (shear-free) jamming occurs.
 \item These jammed states are fragile when applied small shear strees - forces net percolate only in one direction.
 \item When applied certain minimum  shear stress forces netwrok percolates in all directions and system becomes jammed -this transition is controlled 
 by the fracion of force bearing particles which is independent on the density.
 \item On $\phi$ axis in generalized Liu-Nagel diagram there exist two values $\phi_J$ and $\phi_S$. The former is the value at which isotropic jamming occur,
 and below the latter no shear-jammin can occur. In between shear-jamming can occur. 
 \item The value of $\phi_J$ depends on the protocol of jammed system preparation. 
 \item Friction only slightly change the $\phi_J$, however it change number of contacts at jamming, more strongly than expected.
 \item Fragile states - they have a strong force network that carries the majority of the shear stress (deviatoric stress) 
 but which spans the system only in the compressive direction. 
 \item Sshear jammed states - are characterized by two non-parallel populations of force chains.
 \item The fraction of non-rattler grains ($f_{NR}$) controls the percolation of the force networks and emerges as the single parameter distinguishing between unjammed, fragile and shear-jammed states.
 \item Starting from an unjammed state, the strong force network undergoes two sequential percolation transitions controlled by the non-rattler fraction, $f_{NR}$
 \item Contact anisotropy(or stress anisotropy) acts as an order parameter. For values $\psi_J \gg \psi $it does not depend on $\psi$ but if goes to zero at $\psi_J$,
 thus at $\psi=\psi_J$ jammed states become isotropic.
\end{enumerate}


\subsection{The Jamming Transition and the Marginally Jammed Solid\cite{liunagel:10}}
\begin{enumerate}
 \item The simplest possible jamming model: frinctoinless spheres intercationg via repulsive finite-range forces at zero temperature.
 \item The simplest possible model:
 \begin{itemize}
  \item Spheres interact with a potential which vanish at some distance that defines their diamter
  \item At zero temperature, the system is always in mechanical equilibrium
  \item Jammed particles: any infinitesimal force will be resisted by the force network
  \item At a critical packing fraction the system is in between liquid and solid.
  \item The sharp transition exhibits discontinouity as in first-order phase transition, and power-law scaling , as in an ordinary critacal point. 
  \item Zero-temperature transition with aspects of both first- and second- ordeer behaviour, and multiple diverging and vanishing length scales. 
  \item The order parameter that characterize transition is Z, the average number of overlaps a particle has with neighbors. 
  This value jumps discontinuously,,to $Z_c$ at T=0 - because the be in place in mech. eq. it must be surounded by other particles from other sides. 
  \item For frictionless particles $Z_c=2d$. where $d$ is dimensionality.
  \item Below the cutting length: $ \mathrm{l}^{\star}\sim \Delta\phi^{-1/2}$ system looks isostatic, but above it, it's overcoordinated and should behave as a normal elastic solid - can be observed by probing responce to a point force.
  \item Pair-correlation function g(r) diverges at $\phi_J$ with first peak at $r=\sigma$ and amplitude $Z_c$, 
  which aslo diverges as $g(r) \sim \Delta \phi^{-1}$, and $g(r) \propto \sqrt{r-\sigma}$
  \item Numerically calculated exponents are the same in two and three dimensions indicate tahat the transition os mean-field-like. 
  Upper critical dimension of the jamming transition should be d=2.
  \item Each mecahnically stable configuration corresponds to a local minimum in the potential-energy landscape.
  \item Different minima or mechanically stable configurations can have different jamming thresholds, $\phi_C$. 
  The width of distribution vanish in the infinite system size. For 2D disks $\phi_J\approx0.84$ and for 3D spheres $\phi_J \approx 0.64$.
  However, different protocol can provide yield different results, since even in infinity there is still distribution of jamming threashold. 
  \item Jammed spheres at T=0 violate Debey law ($D(\omega)\sim\omega^{d-1}$). Instead $D(\omega)$ is plateau down to zero freq.
  \item The existence of low-freq quasi-localized modes suggests that stability of jammed solids may be different from that of ordinary crystals.
  \item a) Thermal excitations of the same energy gives larger amplitude (for some atoms) for these localized modes - thus making opportunity for breaking contacts.
  \item b) Moreover these low-freq modes have realatively small en. barier to overcome when they are deforming
  \item c) Finally low-participation ratio (mosty low-freq and high-freq modes) are highly unstable to compression.
  \item Different mechanically stable packings created with different initial conditions have the same properties: elastic moduli, coord. number, and other quantities depending on $\Delta \phi$.
 \end{itemize}
 \item Isostatic jammed states can be cleanly studied at T=0, because they are mechanically stable, at configuration which correspond to metastable minimum.
 \item Working at T=0 is appropriate for granular systems and foams, for which energy of even small rearrangement is many order of magnitude greater than thermal flucs.
 \item 
 \item Zero-temperature solid lose its rigidity by creating at least one soft-mode (i.e. mode with zero frequency). 
 In this case the failure of creates another solid (and not a fluid). The complete loss of rigidity should be reflected in behavior
 of just a single mode. The low-freq modes should display characteristic signature of jamming/unjamming. 
 \item Low-freq modes are dominant exctitaions, thus they also control how it responds to a small increase in temperature or applied stress - on top of a decrease in density.
 \item At temperatures greter than zero, a vanishing lenght scale - overlap distance - scales as: $\mathrm{l}_{w} \sim \phi - \phi_{\nu}(T)$, which is consisten with scaling at T=0.
 \item Once rotational degrees of freedom are introduced, the isostatic number in three dimensions increase from $Z_{iso}=6$ to $Z_{iso}=10$.
 However, at the jamming transition $Z_c$ does not jump discontinuously from 6 tp 10, when infinitesimally perturabtion to the shape is introduced. 
 In fact the number of interacting neighbors at the transition increases continously as the shape is varied.
 \item The resolution to the above is that soft modes (freq=0) localized at each sphere, are recruited in non-zero freq. band when deformation is applied.
 At small distortions, $\epsilon$, the resulting rot. modes for a new band that lies below the band of translational modes found for spheres. 
 Then, the onset freq. of the upper bound of modes scales as $\omega^{\star}\sim Z_c(\epsilon) -6$ - exctly the same scaling, that was observed forcompression. 
 \item Frictional spheres:
 \begin{itemize}
  \item Particles have in addition to normal forces $f_n$ also tangential forces $f_t$, which are up to a threashold set by the friction law: $f_t < \mu f_n$.
  \item The presence of tangential forces introduce another equations, thus change counting degrees of freedom.
  \item At the jamming theshold, meechanically stable packings can exist over a range of values: $d+1 \le Z_x \le 2d$.
  \item Contraty to frictionless particles, some properties of the system are not the same - they depend on the way the system was prepared!
  \item However, it is the case that static shear modulus obeys the expected scaling fot J point with $Z_c$ replaced by $Z_{iso}^{\mu}$.
  \item However, packing prepared genlty(and approaching the lowest density that can be typically be accesed for a given $\mu$) 
  tend to have the properties similar to those of frictionless particles sphere packings near the jamming threshold.
 \end{itemize}
\end{enumerate}

\subsection{Computational Modeling of Synthetic Microbial Biofilms \cite{rudge:12}}
\begin{enumerate}
 \item A key factor in the efficiency and robustness of biofilms lies in their spatial organization. 
 \item For a typical colony there can be $10^4-10^5$ cells.
 \item Rigid-body method that includes growth of cells.
 \item In simulations of biofilms, each cell is coupled to many others through biophysical interactions and signaling. 
 \item Since growth occurs on a longer timescale tan biochemical interactions, growth is updated in discrete steps, and intracellular int. and signaling is solved separately.
 \item After each biophysical step, the state is of each cell is updated, and the intracellular and signaling systems are integrated forward by the appropriate time step.
 \item Rod-shaped bacteria, the shape approxitamed by cylider capped with hemispherical ends - capsule.
 \item In TYPICAL growth conditions cells exhibit very little deformation - thus cell lenght can be included as a degree of freedom (and the whole cell keeps the same shape).
 \item A cell is described as $\bar{x}(t) = (c_x, c_y, c_z, \phi_x, \phi_y, \phi_z, L)^T$, where $(c_x, c_y, c_z)$ is cell position, and $(\phi_x, \phi_y, \phi)$ cell orientation.
 Moreover each cell grows at a rate $\dot{L}$ - for small time periods growth is linearized -i.e. constant.
 \item In the low Reynolds number regime appropriate for bacteria, viscous drag dominates inertia - and cells move by distance proportial to applied impulse. 
 \item Upon cell division, a small amount of noise is added to the direction vecotr of each daughter cell - in account of imperfections in cell shape and alignment.
 \item Approximatio of experimentally observed variation in cells' lenght a division threashold is uniformly distribute 3.5-4.5 $\mu m$.
 \item Complex regulatory mechanics that determine cell behavious are sufficiently modeled based on empirical rules. 
 \item Each cell contains set of variables describing its state: position, direction, lenght, radius, volume, area, species, signals, celltype, growthrate, divideFlag. 
 \item Cell signaling is a key part of multicellular organization.  Signaling is introduced to th esimualtion by "medium".
 \item Difficulties in simulating large bacterial populations are: (i) numerical stability of the solution to the simulated system (can be solved with rigid-body dynamics)
 (ii) speed of simulation, which can be solved with e.g. - OpenCl and GPU.
\end{enumerate}


\thispagestyle{empty} % No slide header and footer

\bibliographystyle{unsrt}
\bibliography{article}

\clearpage

\end{document}

%-Cruavzte ssh --progress --no-whole-file --stats --sparse --exclude *.svn*

