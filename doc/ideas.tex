\documentclass[10pt,a4paper]{article}

\usepackage[english]{babel}


\begin{document}

\begin{center}
%\title{ \bf Granular Materials }
%\author{ Pawel Gniewek }
%\maketitle
\textsc{\Large Granular Materials - Biofilms}\\[1.0cm]
\textsc{\LARGE Pawel Gniewek}\\[1.0cm]
\vspace{0.5cm}
\today

\newpage
\tableofcontents
\end{center}

\newpage

\section{Motivation}
\subsection{General information}
\begin{itemize}
 \item Bacteria (and biofils) are everywhere: Dental plaque, hydrothermal vents, tumors
 \item $10^7 - 10^9$ species
 \item Total biomass: $10^{11}$ tons, and $\sim 10^{30}$ cells.
 \item 10x more cells in human body than human cells
\end{itemize}

\subsection{Modeling}
\begin{itemize}
 \item ODE/PDE mass transport and biochemical reactions models
 \begin{itemize}
  \item reaction/diffusion for biomass and substrates
  \item fluid dynamics for the liquid flow
  \item structural mechanics for biofilm growth and EPS
 \end{itemize}
 \item Discrete-element models
  \begin{itemize}
  \item Cellular Automata (rule based)
  \item individual cells
  \item biomass "particles"
 \end{itemize}
 \item Similarity to Granular materials: nails, medicine capsules, colloids and nono-particles, microtubules
 \item Modeling by MD-kind simulations - including size dependent friction.
 \item Cells grow(elongate) exponentailly $ l = l_0 e^{\alpha t}$, and divide when $l > l_m \approx 2l_0$.
\end{itemize}

\subsection{Properties}
\begin{itemize}
 \item Spontaneously formed local nematic order.
 \item Size-, and position- depentend friction leads to streaming instability: fast cell remain small and escape trap, and slow cells grow and slow down even more.
 \item Cell bulcking: Cells grow linealry, and increase longitudal component of stress tensor(Anisotropic stress). 
 Then lateral displacement relieves compression.
\end{itemize}

\subsection{Contact Mechanics}
\begin{itemize}
 \item Central aspects:
 \begin{enumerate}
  \item Pressure and adhesion of acting perpendicular to the contacting bodies' surfaces(normal direction)
  \item Frictional stresses acting tangentially between the surfaces.
 \end{enumerate}

 \item Contact Mechanical Models:
 \begin{enumerate}
  \item Hertz: fully elastic model
  \item Bradley: purely Van der Waals model with rigid spheres
  \item JKR (Johnson, Kendall, Roberts): fully elastic model considering adhesion in the contact zone  - The JKR model considers the effect of contact pressure and adhesion only inside the area of contact. 
  Moreover the model includes histeresis effects - when pulled away they(objects like grains, cells etc.) elongate and stick beyond the distance that corresponds to the sum of their radii.
  \item DMT(Derjaguin-Muller-Toporov): fully elastic, adhesive andVan der Waals model -  contact profile remains the same as in Hertzian contact but with additional attractive interactions outside the area of contact.
 \end{enumerate}

 \item Non-adhesive elastic contacts - normal forces exerted by the surface mainly depend on:
 \begin{enumerate}
  \item depth of indentation(d)
  \item radius of the contact (a)
  \item effective elastic modulus: $\frac{1}{E^{*}} = \frac{1-\nu_1^2}{E_1}+\frac{1-\nu_2^2}{E_2}$, where $E_1,E_2$ are the elastic moduli,
  and $\nu_1,\nu_2$, are Poisson's ratios:
  $\nu = \frac{\mathrm{transverse ~ fractional ~ contraction}}{\mathrm{longitudinal ~ fractional ~ stretch}}$
 \end{enumerate}
 \item The assuption to the Hertzian contact theory:
 \begin{enumerate}
  \item The strains are small and within the elastic limit,
  \item Each body can be considered an elastic half-space, i.e., the area of contact is much smaller than the characteristic radius of the body,
  \item The surfaces are continuous and non-conforming, and
  \item The surfaces are frictionless.
 \end{enumerate}
 When these assumptions are violated, then it leads to problem called non-Hertzian. 
 However for large loads Hertz theory works moslty fine.
 
 \item Adhesive forces introduce changes to Hertzian model:
 \begin{enumerate}
  \item The area of contact was larger than that predicted by Hertz theory,
  \item The area of contact had a non-zero value even when the load was removed, and
  \item There was strong adhesion if the contacting surfaces were clean and dry.

 \end{enumerate}



\end{itemize}



\section{Ideas}
\subsection{Simulations based on Scherck's papers}
\begin{enumerate}
 \item Simulate growing balls with some sort of repulsion: $V = \frac{K}{2}(r-\sigma)^{\alpha}$, where $\alpha=2.5$ is a Hertzian interaction.
 The reasoning behind $\alpha >2.0$ is that when two balls collide and deform one another, 
 the  interface between them, grows faster then the distance separating then gets smaller.
 \item $\alpha=2,\frac{5}{2},0$ - harmonic, Hertzian(in 3D), and hard-spehre repulsion.
 \item Simulate the box with a periodic boundary conditions. However if you want to introduce walls into your box
 do it by the interactions with mirror images (in the respect of the position of the imaginary wall). Be carefull
 around simulating the hole where balls can escape. 
 \item Seed the box with balls' seeds, and then grow them to the desired density.
 Monitor energy of the system. Calculate forces, and then if available relax the system: grow at the beginning and bisecting 
 \item Use different size balls, let's say the radius ration $r_1/r_2 \approx 1.4$
 \item Do attractive completely disrubt the scalling scenerio for the relaxation times that appear near point J for repulsive interactions.
 
\end{enumerate}

\section{Papers}
\subsection{Jamming is not just cool any more\cite{nagel:98}}
\begin{enumerate}
 \item Jammed things: traffic, grainsand, powder raw material.
 \item vibrations from the pounding can reinitiate flow.
 \item When jammed, the disordered system is caught in a small region of phase space with no possibility to escape
 \item Jammed systems are fundamentally different. If the applied stress changes the structure by even a little, the whole jam breaks up. 
 \item One example of a system that is jammed and yet not fragile is foam.
 \item supercooled water, which freezes into glassy state can also be considered as jammed system - sampling noneq. limited space.
 \item Jamming can occure when density is high enough. 
 \item One then can unjam the system by rising the temperature(like in the case of supercooled water), or by applying stress. 
 \item ``Temperature`` - any kind of vibration (mechanical or thermal).
 \item Glass may have a lower glass transition temperature under high shear stress. 
 Likewise, a jammed granular material or foam may have a lower yield stress when random motions (that is, thermal fluctuations) are present.
 \item Conceptual framework from which to investigate the jamming transition, 
 suggesting a possible analogy between temperature and shear stress, the letter being considered alike a thermodynamic parameter.
 \item Examples of three axes:
 \begin{itemize}
  \item A liquid with low viscosity solidifies into a glass when temperature is lowered.
  \item Flowing foam becomes rigid when the applied stress is lowered.
  \item Colloidal suspension loses the ability to flow when the density is increased. 
 \end{itemize}

\end{enumerate}

\subsection{Highly evolved grains \cite{ohern:13}}
\begin{enumerate}
 \item They are far-from-equilibrium systems, as they are too large to experience thermal fluctuations and thus must be externally driven to induce particle motion.
 \item Packings of grains with wishbone shapes stiffen following a concave-up stress-strain curve, 
 and possess failure stresses that are more than a factor of three larger than those for spherical grains.
 \item Other relevant properties: density, shear modulus.Can be optimized as a function of interparticle forces, and packing protocol.
 \item Long-standing open task: identifying the most dilute jammed packings of spherical particles.
\end{enumerate}

\subsection{Repulsive Contact Interactions Make Jammed Particulate Systems Inherently Nonharmonic\cite{schreck:11}}
\begin{enumerate}
 \item When isostatic(in which the number of contacts is the minimum required for mechanical stability) systems lose even a single contact, they become fluidized.
 \item One-side repulsive potential make isostatic jammed system nonharmonic even in the limit of vanishing perturbation.
 \item Isostatic jammed system becomes nonharmonic when even single conact is borke. It occurs above the amplitude $\delta_c$, which average over all exctiation modes $\left<\delta_c\right>$ tends to zero with the number of particles
 \item Interactions potential: $V(r_{ij}) = \frac{\epsilon}{2} \left( 1 - \frac{r_{ij}}{\sigma_{ij}} \right)^2 \Theta\left( 1 - \frac{r_{ij}}{\sigma_{ij}} \right)$, where: $\sigma_{ij} = (\sigma_i + \sigma_j)/2$
 \item The MS(mechanically stable) packings were generated using the compression and energy minimization protocol.
 \item In nonharmonic regime the density of vibrational modes cannot be described using the dynamical matrix ($U_{x,y} = \partial^2 V / \partial x\partial y$)
\end{enumerate}

\subsection{Recent results on the jamming phase diagram \cite{con:10}}
\begin{enumerate}
 \item Soft, frictionless spheres, and jamming defined as a divergence in relaxation time.
 \item At zero temperature and zero shear stress (on the voluem fraction axis i.e. no particle motions), jamming is identified as appearance of mechanical strenght.
 \item The system accuaries mechanical strenth at J-point.
 \item Jamming transition at J-point has mixed first-order-second-order behaviour:
 coexistence of quantities varying continuously at the transition, such as the pressure or the shear modulus, 
 and of quantities changing discontinuously, such as the mean contact number per particle. 
 \item Jamming surface may be also considered as a surface of constant $\tau_g$, where $\tau_g$ is the longest available
 relaxation time to an experiment or simulation - but there is a debate on the shape at $\tau=\infty$.
 \item Some theoretical work suggest that true jamming for soft spheres system can occur only at T=0 (relaxation time diverge only at T=0).
 \item Fluc's in athermal system occur only when system is flowing. When system is jammed no fluc's can ever occur. This is in contrast with the
thermal case, in which fluctuations always occur, even if large fluctuations able to unjam the system may be so rare that they are not observed within any reasonable observation time.
 \item Jamming may occur not at the specific singular poing $\phi$, but in the range $\phi=0.636-0.646$. The reason is that jammed system is
 out-of-equilibrium so the behaviour may depend on the procedure. Thus jamming may occur in narrow, but still segmend of $\phi$.
 \item At $\phi=0.646$ which can be obtained by slow 'equilibrium' compression, because at thaat density pressure diverge.
\end{enumerate}

\subsection{Jamming by shear\cite{bi:11}}
\begin{enumerate}
 \item A jammed system can resist small stresses without deforming irreversibly.
 \item Jamming of frictional, disk-shaped grains can be induced by the application of shear 
 stress at densities lower than the critical value, at which isotropic (shear-free) jamming occurs.
 \item These jammed states are fragile when applied small shear strees - forces net percolate only in one direction.
 \item When applied certain minimum  shear stress forces netwrok percolates in all directions and system becomes jammed -this transition is controlled 
 by the fracion of force bearing particles which is independent on the density.
 \item On $\phi$ axis in generalized Liu-Nagel diagram there exist two values $\phi_J$ and $\phi_S$. The former is the value at which isotropic jamming occur,
 and below the latter no shear-jammin can occur. In between shear-jamming can occur. 
 \item The value of $\phi_J$ depends on the protocol of jammed system preparation. 
 \item Friction only slightly change the $\phi_J$, however it change number of contacts at jamming, more strongly than expected.
 \item Fragile states - they have a strong force network that carries the majority of the shear stress (deviatoric stress) 
 but which spans the system only in the compressive direction. 
 \item Sshear jammed states - are characterized by two non-parallel populations of force chains.
 \item The fraction of non-rattler grains ($f_{NR}$) controls the percolation of the force networks and emerges as the single parameter distinguishing between unjammed, fragile and shear-jammed states.
 \item Starting from an unjammed state, the strong force network undergoes two sequential percolation transitions controlled by the non-rattler fraction, $f_{NR}$
 \item Contact anisotropy(or stress anisotropy) acts as an order parameter. For values $\psi_J \gg \psi $it does not depend on $\psi$ but if goes to zero at $\psi_J$,
 thus at $\psi=\psi_J$ jammed states become isotropic.
\end{enumerate}


\subsection{The Jamming Transition and the Marginally Jammed Solid\cite{liunagel:10}}
\begin{enumerate}
 \item The simplest possible jamming model: frinctoinless spheres intercationg via repulsive finite-range forces at zero temperature.
 \item The simplest possible model:
 \begin{itemize}
  \item Spheres interact with a potential which vanish at some distance that defines their diamter
  \item At zero temperature, the system is always in mechanical equilibrium
  \item Jammed particles: any infinitesimal force will be resisted by the force network
  \item At a critical packing fraction the system is in between liquid and solid.
  \item The sharp transition exhibits discontinouity as in first-order phase transition, and power-law scaling , as in an ordinary critacal point. 
  \item Zero-temperature transition with aspects of both first- and second- ordeer behaviour, and multiple diverging and vanishing length scales. 
  \item The order parameter that characterize transition is Z, the average number of overlaps a particle has with neighbors. 
  This value jumps discontinuously,,to $Z_c$ at T=0 - because the be in place in mech. eq. it must be surounded by other particles from other sides. 
  \item For frictionless particles $Z_c=2d$. where $d$ is dimensionality.
  \item Below the cutting length: $ \mathrm{l}^{\star}\sim \Delta\phi^{-1/2}$ system looks isostatic, but above it, it's overcoordinated and should behave as a normal elastic solid - can be observed by probing responce to a point force.
  \item Pair-correlation function g(r) diverges at $\phi_J$ with first peak at $r=\sigma$ and amplitude $Z_c$, 
  which aslo diverges as $g(r) \sim \Delta \phi^{-1}$, and $g(r) \propto \sqrt{r-\sigma}$
  \item Numerically calculated exponents are the same in two and three dimensions indicate tahat the transition os mean-field-like. 
  Upper critical dimension of the jamming transition should be d=2.
  \item Each mecahnically stable configuration corresponds to a local minimum in the potential-energy landscape.
  \item Different minima or mechanically stable configurations can have different jamming thresholds, $\phi_C$. 
  The width of distribution vanish in the infinite system size. For 2D disks $\phi_J\approx0.84$ and for 3D spheres $\phi_J \approx 0.64$.
  However, different protocol can provide yield different results, since even in infinity there is still distribution of jamming threashold. 
  \item Jammed spheres at T=0 violate Debey law ($D(\omega)\sim\omega^{d-1}$). Instead $D(\omega)$ is plateau down to zero freq.
  \item The existence of low-freq quasi-localized modes suggests that stability of jammed solids may be different from that of ordinary crystals.
  \item a) Thermal excitations of the same energy gives larger amplitude (for some atoms) for these localized modes - thus making opportunity for breaking contacts.
  \item b) Moreover these low-freq modes have realatively small en. barier to overcome when they are deforming
  \item c) Finally low-participation ratio (mosty low-freq and high-freq modes) are highly unstable to compression.
  \item Different mechanically stable packings created with different initial conditions have the same properties: elastic moduli, coord. number, and other quantities depending on $\Delta \phi$.
 \end{itemize}
 \item Isostatic jammed states can be cleanly studied at T=0, because they are mechanically stable, at configuration which correspond to metastable minimum.
 \item Working at T=0 is appropriate for granular systems and foams, for which energy of even small rearrangement is many order of magnitude greater than thermal flucs.
 \item 
 \item Zero-temperature solid lose its rigidity by creating at least one soft-mode (i.e. mode with zero frequency). 
 In this case the failure of creates another solid (and not a fluid). The complete loss of rigidity should be reflected in behavior
 of just a single mode. The low-freq modes should display characteristic signature of jamming/unjamming. 
 \item Low-freq modes are dominant exctitaions, thus they also control how it responds to a small increase in temperature or applied stress - on top of a decrease in density.
 \item At temperatures greter than zero, a vanishing lenght scale - overlap distance - scales as: $\mathrm{l}_{w} \sim \phi - \phi_{\nu}(T)$, which is consisten with scaling at T=0.
 \item Once rotational degrees of freedom are introduced, the isostatic number in three dimensions increase from $Z_{iso}=6$ to $Z_{iso}=10$.
 However, at the jamming transition $Z_c$ does not jump discontinuously from 6 tp 10, when infinitesimally perturabtion to the shape is introduced. 
 In fact the number of interacting neighbors at the transition increases continously as the shape is varied.
 \item The resolution to the above is that soft modes (freq=0) localized at each sphere, are recruited in non-zero freq. band when deformation is applied.
 At small distortions, $\epsilon$, the resulting rot. modes for a new band that lies below the band of translational modes found for spheres. 
 Then, the onset freq. of the upper bound of modes scales as $\omega^{\star}\sim Z_c(\epsilon) -6$ - exctly the same scaling, that was observed forcompression. 
 \item Frictional spheres:
 \begin{itemize}
  \item Particles have in addition to normal forces $f_n$ also tangential forces $f_t$, which are up to a threashold set by the friction law: $f_t < \mu f_n$.
  \item The presence of tangential forces introduce another equations, thus change counting degrees of freedom.
  \item At the jamming theshold, meechanically stable packings can exist over a range of values: $d+1 \le Z_x \le 2d$.
  \item Contraty to frictionless particles, some properties of the system are not the same - they depend on the way the system was prepared!
  \item However, it is the case that static shear modulus obeys the expected scaling fot J point with $Z_c$ replaced by $Z_{iso}^{\mu}$.
  \item However, packing prepared genlty(and approaching the lowest density that can be typically be accesed for a given $\mu$) 
  tend to have the properties similar to those of frictionless particles sphere packings near the jamming threshold.
 \end{itemize}
 \item  Boson peak - it seems that it emerges as a result of appearance of new exctitations upon jamming.
 One of the property of boson peak exctitations is that these exctitation are poor transporters of energy.
 \item The ratio $C(T)/T^3$ of the heat capacity to the expected $T^3$ dependence (predicted by Debye model for xtalline solids), 
 exhibits a characteristic peak - known as boson peak.
 \item Anharmonic effects in jammed spheres packings are quite different from those in crystals. 
 \item Moreover, the low-frequency(and quasi-localized) modes of jammed systems are, considerably more anharmonic than the high-frequency ones. 
\end{enumerate}

\subsection{Computational Modeling of Synthetic Microbial Biofilms \cite{rudge:12}}
\begin{enumerate}
 \item A key factor in the efficiency and robustness of biofilms lies in their spatial organization. 
 \item For a typical colony there can be $10^4-10^5$ cells.
 \item Rigid-body method that includes growth of cells.
 \item In simulations of biofilms, each cell is coupled to many others through biophysical interactions and signaling. 
 \item Since growth occurs on a longer timescale tan biochemical interactions, growth is updated in discrete steps, and intracellular int. and signaling is solved separately.
 \item After each biophysical step, the state is of each cell is updated, and the intracellular and signaling systems are integrated forward by the appropriate time step.
 \item Rod-shaped bacteria, the shape approxitamed by cylider capped with hemispherical ends - capsule.
 \item In TYPICAL growth conditions cells exhibit very little deformation - thus cell lenght can be included as a degree of freedom (and the whole cell keeps the same shape).
 \item A cell is described as $\bar{x}(t) = (c_x, c_y, c_z, \phi_x, \phi_y, \phi_z, L)^T$, where $(c_x, c_y, c_z)$ is cell position, and $(\phi_x, \phi_y, \phi)$ cell orientation.
 Moreover each cell grows at a rate $\dot{L}$ - for small time periods growth is linearized -i.e. constant.
 \item In the low Reynolds number regime appropriate for bacteria, viscous drag dominates inertia - and cells move by distance proportial to applied impulse. 
 \item Upon cell division, a small amount of noise is added to the direction vecotr of each daughter cell - in account of imperfections in cell shape and alignment.
 \item Approximatio of experimentally observed variation in cells' lenght a division threashold is uniformly distribute 3.5-4.5 $\mu m$.
 \item Complex regulatory mechanics that determine cell behavious are sufficiently modeled based on empirical rules. 
 \item Each cell contains set of variables describing its state: position, direction, lenght, radius, volume, area, species, signals, celltype, growthrate, divideFlag. 
 \item Cell signaling is a key part of multicellular organization.  Signaling is introduced to th esimualtion by "medium".
 \item Difficulties in simulating large bacterial populations are: (i) numerical stability of the solution to the simulated system (can be solved with rigid-body dynamics)
 (ii) speed of simulation, which can be solved with e.g. - OpenCl and GPU.
\end{enumerate}

\subsection{Diffusion-Limited Growth in Bacterial Colony Formation \cite{MH:90}}
\begin{enumerate}
 \item When biological grotwh is governed by diffusion-limited processes, growing patterns show features such as screening, repulsion etc.
 \item Growth of bacterial(Bacillus subtilis) colonies on agar plates is governed by DLA processes contained in plates.
 \item Bacterial colonies growth is affected by environmental physical conditions and the way bacteria prolifirate. 
 \item On plates, with nutritients concentration 1g/L colonies develop in fractal structure with a dimension $\approx 1.72$ - obtained from box counting method.
 \item Branches screening effects observed - characteristic to pattern formation in Laplacian field.
 \item Two colonies spearated by small distance repeal each other, and neve fuse - characteristic for Laplacian field. 
 \item The Laplacian field is made by diffusin nutritients and not by some kind of growth-blocking metabolic waste. 
 \item In the limit of low nutrient concentrations the growth of bacterial colonies is generally controlled by diffusion-limited processes.
 \item DLA consists generally from two mechanisms: (i) Laplacian field and (ii) local growth mechanics. In case of bacterial colonies these are:
 (i) nutritien concentration and (ii) bacterial proliferation. Moreover, for too high nutrients concentration DLA pattern is not fomred -  diff. is not limiting factor any longer. 
\end{enumerate}

\subsection{Vibrations of jammed disk packings with Hertzian interactions \cite{schreck:13}}
\begin{enumerate}
 \item Dry granular media - discrete grains, interact via purely repulsive, frictional contact interactions.
 \item Sources of nonlinearity in dry materials include:
 \begin{itemize}
  \item the nonlinear form ofHertzian interactions betwen grainsans
  \item conact breaking and formation
  \item dissipation
  \item rolling and sliding frictional contacts
 \end{itemize}
 \item This manuscript concerns nonlinearitis that arise from the shape of Hertzian interactoins.
 \item (i) The shape of Hertzian interaction give rise to nonlinearities in the vib. response of jammed disk packings.
 Right after the first contact is broken the system creates set of discrete harmonics and beats among these and normal mode freqs, 
 unitl the moment where on average 1 conact is borken - then creates a continous spectrum of vibrations. 
 \item (ii) These nonlinearities are weak compared to those generated by contact breaking.

\end{enumerate}

\subsection{Localized excitations in a vertically vibrated granular layer \cite{swinney:96}}
\begin{enumerate}
 \item Oscillons - stable,localized 2D exctiations.
 \item Granular standing waves is a cooperative behaviour resulting from coolisions between grains. 
 Those waves can from stripes, squares, hexagonals etc. 
 \item Oscillons emerge under the same conditions as standing waves.
 \item An oscillon is a small circullar simetric excitation w/ a freq f/2 (f - freq of shaking the plate on which grains are placed).
 \item Oscillons are long lived objects. Moreover, since they are subharmonic craters and peaks can coexsist. 
 \item Oscillons do not propagate, but they drift slowly - they need about 1000 cycles to drift the distance of one diameter. 
 \item Oscillons of the same phase show short-range repulsions, and of the opposite phase-short range attration. 
 \item With the above "interactions" when they collide, they create stable structures: diploe, polumer, trimer, ion-lattice - Coordination numbers higher than 3 are not creater - for an isolated structre.
 \item Plannar patters can be considered in some cases as collection of oscillons. 
 \item Oscillons form in narrow range of frequencies, when dissipation and hysteresis are large. 
\end{enumerate}


\subsection{A cell-based simulation software for multi-cellular systems \cite{dd:10}}
\begin{enumerate}
 \item Multi-cellular systems are of multi-scale nature, and the research is shifting towards studies of whole cells or populations of cells
 \item Several milions of cells (in 3D) are simulated by  agent-based models.
 \item Each cell is modeled as an isotropic, elastic, and adhesive sphere.
 \item Each cell is capable of migration, growth and division.
 \item Cell-cell and cell-matrix is modeled by JKR model (validated by Chu et. al, 2005).
 \item Cells are parametrized by: 
 \begin{enumerate}
  \item Young's modulus, 
  \item Poisson ratio
  \item the density of membrane receptors -  responsible for cell-cell and cell-substrate adhesion 
  \item its (intrinsic) radius
 \end{enumerate}
 \item Prolifireting cells double their sizes, and deform into dumb-bells shape, before splitting into two daughter cells (Drasdo'95).
 When the exerted pressure is $p_tot > p_0$ pressure then cell goes into rest phase G0. The orientation of the cell divisio is random.
 
 \item Cell migration is modeled by Langevin equations: summarizing JKR force, friction forces (c-c, c-matrix: parametrized by friction coeff.), and optionally random forces.
 \item Equation for diffusion and consumption of nutrients, growth factors, etc. 
 on a lattice are solved using the Euler forward method 
 (fulfilling the Courant-Friedrichs-Lewy condition i.e. time step must be smaller than the value providing numerical stability of the solution)
 \item Comparsions between JKR, Hertz and DMT models suggests that the  precise form of the interaction force may have no significant impact on the multi-cellular dynamics.
 \item In the absence of chemotactic signals, isolated cells in suspension or culture medium have been observed to perform an active random-walk-like movement(Gruler, 94; Glazier'96).
 \item Cells parameters: diameter, diff const, energy of single adhesive bond, cell fric. const, cycle time, max. cycle time variation, Young modulus, densitiy of adhesion molecules. 
\end{enumerate}

\subsection{Johnson-Kendall-Roberts Theory Applied to Living Cells \cite{pin:05}}
\begin{itemize}
 \item When contacting surfaces adhere weakly and deform a little, then DMT approach works.
 \item At higher adhesion and deformability, when surfaces are upon separating forces, 
 there is a finite, nonzero contact area at separationt - then JKR theory ives the relation between the pull off force $F_s$ and the adhesion energy $W_{adh}$
 \item When the cytoskeleton of the cells has a complete 3D structure that maintains a slightly deformable spherical shape, 
 JKR theory is applicable to relate the separation force to the adhesion energy. In this case, when cytoskeleton is responsible for
 cells' spherical shape, the cells do not behave like shells but like elastic spheres.
 \item Adhesion(i.e. attraction) can be induced for example by depletion effect cause by nonadsorbing, water-soluble polymers.
 \item Model was tested JKR theory based on ahdesion forces when pulling away two cells. 
 \item Elastic behavior is expected up to the time, when cytoskeleton cannot reshape itself upon the force load (plastic flow, which take up to several minutes).
 \item 
\end{itemize}

\subsection{Mechanically Driven Growth of Quasi-Two-Dimensional Microbial Colonies \cite{bw:13}}
\begin{enumerate}
 \item The strength of mechanical interactions determines the speed with which the colony expands in space, with diffusion of the nutrient playing a secondary role.
 \item Cells are modeled as growing spherocylinders with a variable length.
 \item The colony grows on a 2D flat surface with nutrient concentration c(x,y)( diffusion with a diff. coeff. D).
 \item Nutrients are consumed at a rate $kf(c) = k\frac{c}{c_{half}+c}$ per unit biomass density, with half-saturation constant $c_{half}$.
 \item Cells grow (by elongation) at a rate $v_gf(c)$. 
 \item The cells interact mechanically  by the Hertzian theory of elastic contact.
 \item the nutrient becomes depleted within the colony so that only cells in a thin layer at the front are growing.
 \item nonmotile microorganisms replicate and push each other away as they grow.
 \item Mechanical interactions are understood to be very important in such systems; 
 in particular, mechanical pressure has been hypothesized to strongly affect the growth and apoptosis rates of cells, 
 leading to an alternative form of growth limitation.
 \item The density profile close to the edge decays according to a power law towards the uncompressed cell density $\rho_0$. 
 It is in striking contrast to Fisher-Kolmogorov waves, which exhibit exponential density profiles in the wave tip.
\end{enumerate}

\subsection{Buckling instability in ordered bacterial colonies \cite{tsimring:11}}
\begin{enumerate}
 \item Complex spatio-temporal organization of bacterial colonies may depend on:
 shape, motility, membrane structure, hemotaxis, cell,cellcommunication, and direct mechanical contact among cells in dense colonies.
 \item In biofilms or colonies in cavities, long-range signaling may be of secondary importance. 
 \item in close proximity, such as in biofilms, bacteria usually lose their flagellae and become non-motile.
 \item In large colonies the perfect nematic order is never reached; 
 multiple domains with different orientations are constantly regenerated in the bulk of the colony.
 \item Growing colony of nematically ordered cells is prone to the buckling instability - destroying  perfec nematic order. 
 This instability is related to the anisotropy of the stress tensor in the ordered cell colony.
 \item Buckling instability occurs at the place where the density of cells and the pressure are greatest.
 \item In continuum model granilarity of cells has been neglected, and cells were modeled as continous fileds: density $\rho(\mathbf{r})$ and velocity $\mathbf{v}(\mathbf{r})$.
 \item When cells are densely packed, the 'cellular fluid' is incompressible.
 \item Discrete-elements simulation:
 \begin{itemize}
  \item Each cell is represented by a spherocylinder with fixed diameter d and variable length l.
  \item Cells grow exponentailly unitl they reach critical length, at which every cell is replaced by two daughter cells.
  \item The cricital lenght is assigned randomly at cell's birth -  to break artifical synchornisation of cells division.
  \item The normal and tangential (frictional) forces moving the cells are computed based on the overlap of virtual soft spheres centered at the nearest points on the  axes of interacting spherocylinders.
  \item It is expected that the buckling instability may strongly affect the structure of bacterial populations 
  in confined environments (such as surface-bound biofilms), if significant internal stresses develop there due to the cell growth.
  \item Buckling istability may also affect the tissue growth and structure in multicellular organisms 
 \end{itemize}

\end{enumerate}

\subsection{Particle-Based Multidimensional Multispecies Biofilm Model \cite{pic:04}}
\begin{enumerate}
 \item Processes ongoing in biofils:
 \begin{enumerate}
  \item growth and decay
  \item division and spreading
  \item substrate transport and reactions
  \item biomass detahment
  \item liquid flow past the bioflm
  \item attachment
 \end{enumerate}
 \item Model:
 \begin{enumerate}
  \item Reactor contains two compartments: bulk liquid(completely mixed), and biofilm.
  \item Biomass consists of active(up tp \#B types) and inert particles(that result of the decay of active particles).
  \item Particle may represent an individual cell or cluster of cells.
  \item Particles are hard-spheres.
  \item The growth of every particle depends on accessible nutrients.
  \item The total mass in spherical particles is limited by $M_{max}$, which is independent of particle's type.
  \item When max mass limit is reached, new daughter particle is created, which touches mother particle in a random position, 
  and part of mother's mass is redistributed among daughter particle.
  \item PBC are used, and particles are spreading as a result of exerted forces by other particles.
  \item Concentration field is solved numerically including reaction(nutr. consumption) component.
 \end{enumerate}
 \item Investigated properties:
 \begin{enumerate}
  \item spatial dist. of substrate concentration and fluxes
  \item biofilm structure
  \item total biofilm mass
  \item dynamics of biomass distr.
  \item dynamics of biofilm detachment
  \item dynamics of solute concentration
  \item steady-state concentration of biomass and solutes in the biofilm
 \end{enumerate}
 \item A perfect steady-state with respect to the spatial dist. of particles and solutes is not reached.
 It is mainly due to stochasticity in the model, which comes from:
 \begin{enumerate}
  \item The initial distribution of biomass particles on the surface
  \item random choice of direction for the placement of a daughter cell
  \item uneven cell division
 \end{enumerate}
\end{enumerate}

\subsection{A single-cell-based model of tumor growth in vitro: monolayers and spheroids \cite{dd:05}} 
\begin{enumerate}
 \item Types of models:
 \begin{enumerate}
  \item single-cell-based (including cellular-automata, and off-latice)
  \item continuum models
 \end{enumerate}

 \item The model:
 \begin{enumerate}
  \item Basic unit of the model is a signle cell. Each cell is an elatic, sticky particle of limited compressibility and deformability.
  It's also capable of active migration, growth and division.
  \item During the divisin phase(mitosis) the cell has dumb-bell shape.
  \item Attractive and repulsive interactions are modeled using JKR model.
  \item Cells perform RW (in an absence of chemotactic signals) - characterize by diffusion constant D
  \item Friction-dominated overdamped stochastic motion is modeled with Metropolis algorithm.
  \item Glucose is limiting nutritent for the model of tumor spheroids in suspention.
  Cells can grow only if the local glucese contentration is above a certain threshold, and die by necrosis bolow other conc. threshold.
  \item Glucose is loally consumed with the rate $\hat{\gamma}$ and is described by partial-differential equation describing reaction-diffusion kinetics.
  \item The influence of extracellular matrix on cells migration is neglected.
  \item Average cell cycle time is influenced by nutritiens, regulatory factors, and mechanical stress.
  \item Cell can grow only when if it's not extensively deformed (or compressed) - i.e. the growth of the cell may be controlled by the compression sensed by cell's cytoskeleton.
  \item Three scenerios are possible:
  \begin{enumerate}
   \item Cell grows when a deformation is removed.
   \item A cell that was subjected to the critical deformation longer then some threashold does not enter the cell cycle again.
   \item If the critical deformation lasted longer than some specified time the cell undergoes apoptosis.
  \end{enumerate}

  \item Lysis of dead celly is modeled as a very fast or very slow process by: rmoving or not removeing at all a dead cell.
 \end{enumerate}
 \item Division of nonboundary cells(in tumer monolayer cells) is repressed by the neighbor contacts.
 \item The highest proliferation actvity is close to the tumor boundary - where the local concentration of glucose is the highest, 
 while inside the tumor a necrotic core forms.
 \item If at the same time each cell is supplied with glucose then cells divide everywhere hence the tumor grows exponentially fast.
 \item Simulations suggest that tumor grows exponentailly fast, but later is followed by a linear expansion of the tumor diamteres - both in layer and spherodi case.
 \item It is suggested, that linear growth regime is characterized by a cell number increase confied to a surface layer. 
 A nutritient limitation in tumor spheroids affects slighlty the expansion velocity.
 \item In case cells are of the average size and in absence of apoptotic or nectoric core, the linear expansion regime is accompanied by power law $N \propto t^d$.
 \item Cells at tumor boundary have performed more cell divisions and accumulated more mutation than cells in the interior, or in cells of a tumor of equal 
 size 
 \item Above a certain glucose concentration the growth kinetic in tumor spherodis should become independent of the glucose conc. and be determined by biochemical interactions only.
 \item The extend of exponential growth depends on cessation of cell division conditions.
 \item Increase of cell mobility (lower viscosity etc.) extends exponential growth phase, and increase the growth velocity.
 \item In case of cell-cell adhesion absence, tumor initially grows exponentailly fast,  while the aggregate diameter grows as $R \propto \sqrt{t}$ and then corss into linear regime
\end{enumerate}


\subsection{Dissipative particle dynamics simulations for biological tissues: rheology and competition \cite{elgeti:11}}
\begin{enumerate}
 \item Mechanical stresses couple to the orientation of the mitotic spindle during cell division (Julicher'07).
 \item Pressure strongly influences the expression of some genes (Farge'08).
 \item Cell differentiation depends on the mechanical properties of the substrace - in particular th value of its elastic modulus.
 \item Nonlinear elastic description of tissue has been very succesful.
 \item At the early stage of development, many tissues should be considered as liquids (finite viscosity and isotropic surface tension).
 \item The model:
 \begin{enumerate}
  \item Cell must reach some specified size, in order to enter mitosis stage - when it's reached (i.e. the distance between two particles creating a cell - see below) then cell divides
  \item Each cell is modeled as two particles interacting via a repulsive growth force - a distance between this two particles represent cell's size.
  \item Apoptosis acts randomly, removing cells with a constant rate $k_a$.
  \item Forces are divided into attractive and repulsive.
  \item Equations of motion are integrated with Dissipative Particle dynamics (Nikunen'03).
  The advantage oevr Langevin dynamics is that DPD does not assume  the background dissipation to be the dominant mode of dissipation - e.g. dissipative forces may be transfered to neighboriing cells.
  \item MY NOTE: alternatives to DPD are: fluctuating lattice-Boltzmann method(Ladd), or Stochastic Rotation Dynamics (Kapral). 
  A good description of DPD is in "Understanding Molecular Simulations, 2nd Ed., Chpt:17"
  \item Cell division and apoptosis break momentum balance. In that case a small background dissipation is needed.
  \item Particles belonging to the same cell act with repulsive potetnail. 
  Volume exclusion is modeled with a repulvie forces $\sim r^{-5}$ up to some cutoff value, and adhesion is modeled as a linear force. 
 \end{enumerate}
 \item Adhesive forces are at the origin of surface tissue tension that cause tissue aggregation an d round up.
 \item Cell division and apoptosis lead to fluidizaion of the tissue. 
\end{enumerate}


\subsection{Mechanical Control of Cell flow in Multicellular Spheroids \cite{dm:13}}
\begin{itemize}
 \item In embrology, collective migration occurs during morphogenetic transformations, in the absence of any cell proliferation.
 \item Cell migration is observed in tumors: in response to chemical cues, cancer cells can escape the primary tumor, and invide adjacent tissues. 
 The process is either collective and single cell.
 \item In tumors and spheroids, gradients of oxygen, grow factors etc. lead to increase of cell proliferatoin at the periphery, and apoptosis in the center, which drives cells flow.
 \item Growt rate of multicellular spheroids is drastically reduced by an external mechanical stress (Capello'11, '12), and this effect saturates at 5000Pa
 \item Cell division, rather than cell death or density, is affected by the mechanical stress.
 \item In steady-state spheroids, in the absence of mechanical stress cells flow toward the center is observed. 
 This convective motion is induced by cell division. 
 \item In growin spheroids, in the absence of stress moving back and forth toward the center was obseerved.
 It suggests that there is convergend flow, and divergent flow as well, but under the mechanical stress divergent flow is impeded. 
 \item The transport model:
 \begin{enumerate}
  \item Particles(markers) flux is: $J = \nu \rho$, where $\nu $ is the cell velocity, and $\rho $ is markers density
  \item diffusion of markers is neglected (they don't leave prolifiratnig cells)
  \item Prolifireting cells diffusion contribute to the displacement as $\mathrm{d}/\mathrm{R}$, 
  where $\mathrm{d}$ is cell's diameter and $\mathrm{R}$ is spheroid radius.
  \item Chemotaxis is also neglected
  \item Transport equation is: $\partial_t \rho + \nabla \nu \rho = 0$
 \end{enumerate}
 \item Growth rate decreased by 30\% under mechanical stress.
 \item The mode, which is involving purely convection fluxes, is in very good agreement with the experiment, 
 what suggests that for this system chemotaxis and diffusion are indeed negligible.

 
\end{itemize}


\thispagestyle{empty} % No slide header and footer

\bibliographystyle{unsrt}
\bibliography{article}

\clearpage

\end{document}

%-Cruavzte ssh --progress --no-whole-file --stats --sparse --exclude *.svn*

